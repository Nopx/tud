\documentclass{sig-alternate-05-2015}
\usepackage{endnotes}
\let\footnote=\endnote

\begin{document}


\title{Business Idea: Version Control Software for Video Data}
\author{Bernard Jollans}
\maketitle

\section{Explanation}
My entrepreneurial idea is a version control software for video data. Version control is an essential part of software development and many other text based things. It helps in tracking your changes and reverting back to them in your project. It is also widely used to enable multiple actors to work on the same project at the same time. Software like Git and SVN are not only used by the majority of companies, but also get introduced to students in university. This nearly total integration into the industry shows, that version control is a useful tool on which you can rely for organization.
The reason why you cannot use Git or SVN effectively for things like images, videos and other media files is, that in order to save two versions of a project, both versions have to be saved completely. In text based projects it is enough to save the most current version together with differences, with which old versions can be recreated. 
Seeing that memory does not come for free, it would save more than just filepace to have similar technology for everything. As different file types behave very differently though, this idea focuses on the kind of projects with the biggest industries using them. Namely video projects.

\section{Entrepreneurial Mindset}
Since developing my idea is a long shot into a complex field, determination is very important. The idea not only sounds tough, but may be impossible. It is therefore important to keep the end goal in mind, when partners or others start doubting the idea\footnote{How does individualism influence your decision-making? }. I also believe, that with this business idea there is more to be gained than money. Managing to develop such a version control system would be a substantial achievement and would elevate my self perceived worth\footnote{What roles does need for achievement play in your decisions? }.
Even though the technology is not yet developed and its finalization seems far away, I believe, that anything is possible if you have enough time. Knowledge can be acquired, 

\section{Maroeconomic Change}
It has never been easier to make a video. It has also never been easier to watch it. The growth of the internet and with it the growth of youtube have given an extra boom to the industry, which once only consisted of Hollywood. Everybody can now make a video and watch it on demand, with the phone in their pocket. Videos as a medium have become ever more important and normal over the last decade.
This has caused the global amount of video data to increase exponentially. Even though memory technology has also steadily been getting better and cheaper, it cannot keep up. This situation helps in creating the problem of big data.
A version control software for video data would help reduce this by a lot. Multiple versions of the same video, would not clutter up ones memory anymore. Because every video (on Youtube or from Hollywood) exists in multiple versions before its release, this technology would reduce video data as a whole.

\section{Industry Condition}
The film/video industry as a whole is huge and growing rapidly. Selling to producers in this industry opens a mass market. Entering this market gives us competitor giants like Adobe, Microsoft and Sony who all have great experience in developing video editing software. This experience is not easy to acquire, as video codec standards are long and complicated. Luckily, knowledge is pretty much the only thing required to enter this market. 

\section{Industry Status}
The film / video industry is steadily growing as a whole. Even though growth of the U.S. film industry is slowing down, new markets are emergin, like China and France. A potentially grave outlook, is the rising popularity of virtual reality. The video industry is rather old by now and may be reaching its peak soon. Developing a version control software for normal videos only to find them being replaced by VR-videos, may cost a lot. However the additionally needed development will not be too great, if the software is developed in a scalable and adaptable manner.

\section{Competition}
The competition is the biggest issue concerning this idea. To enter the market, the version control software should be integrated into a video editing application and the producers of these applications are gigantic sopisticated companies, like Adobe, Microsoft and Sony. Competing agaonst these companies would be very tough and costly. It would not only be difficult to develop similarly sophisticated video editing software, but also to gain customers. Since learning how to use video editing software is a long process and not all software is the same, users are often reluctant to change.
A simple way around this conundrum is to work together with existing companies. The version control system can either be integrated into the software or later added as a plugin. This would change the competitors to other third party developers for video editing software.

\section{Value Innovation}
The version control system will be integrated into existing video editing tools. The user has the choice between a software with our version control or the same software without it. Assuming, that our version control does not cause a big overhead, no value will be lost to the customer. However value will be gained. First of all the amount of needed disk space will be reduced, thus reducing the overall cost of editing. It also eases organization of video data, potentially increasing the editing speed. On top of that the editor has the opportunity to use an older version, which he would otherwise not have cared to save. In total the video data version control system only gives advantages to the user.
It would be possible to compare my idea to existing additions to video editing software. This however would always require a specific case analysis, since these additions differ strongly in what they do and how they do it.

\section{Opportunity Identification}
Making films / videos generates a lot of data. This brings up many issues, the most obvious one is the storage. For every bit that memory is becoming cheaper, videos use more data. The amount of daily generated videos is steadily increasing and the videos are getting bigger in size. The standard resolution has gone up from 1k to 2k in no time. In many commercials 4k is already beeing advertized. Additionaly, with the growing popularity of 3D-Cinema, the value of saved data doubles. This mass in data for the film industry alone makes organization ever harder.
A version control software would promise to take care of both of these issues, saving the film industry a lot of money.

\section{Building a Team}

\theendnotes
\end{document}
