\documentclass[a4paper]{article}

\usepackage{endnotes}

\let\footnote=\endnote

\renewcommand{\notesname}{Questions}
\setcounter{secnumdepth}{0}

\begin{document}

\title{Business Idea: Version Control Software for Video Data}
\author{Bernard Jollans}
\maketitle

\section{Explanation}
My entrepreneurial idea is a version control software for video data. Version control is an essential part of software development and many other text based things. It helps in tracking changes and reverting back to them. It is also widely used to enable multiple actors to work on the same project at the same time. Software like Git and SVN are not only used by the majority of companies, but also get introduced to students in university. This nearly total integration into the industry shows, that version control is a useful tool on which you can rely for organization.
The reason why you cannot use Git or SVN effectively for things like images, videos and other media files is, that in order to save two versions of a project, both versions have to be saved completely (only for non text based files). In text based projects it is enough to save the most current version together with differences, with which old versions can be recreated. 
Seeing that memory does not come for free, it would save more than just filespace to have similar technology for everything. Because different file types behave very differently, this idea focuses only on the kind of projects used by the biggest industries. Namely video projects.
A version control software for videos would mainly: reduce required memory, help organization of large video data sets, help multiple people to work on the same project and enable the option to revert back to old changes easily\footnote{What key customer values does your solution offer?}.

\section{Entrepreneurial Mindset}
Since developing my idea is a long shot into a complex field, determination is very important. The idea not only sounds tough, but may be impossible. It is therefore important to keep the end goal in mind, when partners or others start doubting the idea\footnote{How does individualism influence your decision-making? }. I also believe, that with this business idea there is more to be gained than money. Managing to develop such a version control system would be a substantial achievement and would elevate my personally perceived worth\footnote{What roles does need for achievement play in your decisions? }.
Even though the technology is not yet developed and its finalization seems far away, I believe, that with enough time, anything is possible. I therefore believe that I have complete control over if this idea will turn out well\footnote{Do you have an internal and external locus of control?}. Knowledge can be acquired easier than ever before and knowledge is the only thing needed for this project. I know I will remember this when it seems like it cannot be done. The danger of such optimism is always overconfidence, which seems like the biggest problem I will face\footnote{Can you be optimistic, while managing the risks of overconfidence?}.

\section{Entrepreneurial Motivation}
As mentioned above, I strongly believe, that I can make the development of this technology possible without having to rely on any one person. I will however have to rely on a team. The team might change over time and I am able to recruit new members, but I will not be able to develop such a groundbreaking technology on my own. However I will not make the mistake of burdening my team with the responsibilities that I should carry.\footnote{Do you have high self-efficacy?}
The tolerance for ambiguity needed for this project is not very high, since its stakes are not very high. In this project the only money that will have to be paid at first, are the salaries of my team. This overall compared is very little\footnote{What is your tolerance for ambiguity?}. 
The main driver of this project will be my cognitive motivation. I am fascinated by the subject of my idea. I think video processing is one of the most impressive fields you can work in, not only because it is hard, but also because the knowledge required is so technical.\footnote{Do you exhibit high cognitive motivation?} Since my current studies do not go exactly into this direction, I will have a harder time than some while developing this. Yet I am studying with many others, who have chosen to focus on media technology. My network at TUDelft is so far being dominated by people who will understand the algorithms in video codecs and the underlying ideas\footnote{How can your network help you?}.

\section{Maroeconomic Change}
It has never been easier to make a video. It has also never been easier to watch it. The growth of the internet and with it the growth of Youtube have given an extra boom to the industry, which once only consisted of Hollywood. Everybody can now make a video and watch it on demand, with the phone in their pocket. Videos as a medium have become ever more important and normal over the last decade.
This has caused the global amount of video data to increase exponentially. Even though memory technology has also steadily been getting better and cheaper, it cannot keep up. This situation helps in creating the problem of big data. But it is not only the amount of videos that increases data. Film studios are trying to deliver content in ever better quality. 1080p went to 2k and is now at 4k. The evolution of 3D-Cinema has made it mandatory for many studios to film with two cameras and have double the data\footnote{What demographic and psychographic changes are creating new market needs in business 
areas that interest you? }\footnote{What technical advancements, political changes, and/or regulatory changes are creating new market opportunities in business areas that interest you? }.
A version control software for video data would help reduce this by a lot. Multiple versions of the same video, would not clutter up memory anymore. Because every video (on Youtube or from Hollywood) exists in multiple versions before its released, this technology would reduce video data as a whole\footnote{What existing factors can you eliminate?}.

\section{Industry Condition}
The film/video industry as a whole is huge and growing rapidly. Selling to producers in this industry opens a mass market. Entering this market gives us competitor giants like Adobe, Microsoft and Sony who all have great experience in developing video editing software. This experience is not easy to acquire, as video codec standards are long and complicated. Luckily, knowledge is pretty much the only thing required to enter this market. 
If the development of my idea succeeds, we will have a knowledge advantage, which consists of the technology behind our version control software. This will give us the edge we need over our competitors\footnote{What knowledge do you possess that can contribute to serving a market need within your industry of interest? What are the demand conditions?}.

\section{Industry Status}
The film / video industry is steadily growing as a whole. Even though growth of the U.S. film industry is slowing down, new markets are emerging, like China and France. A potentially grave outlook, is the rising popularity of virtual reality. The video industry is rather old by now and may be reaching its peak soon. Developing a version control software for normal videos only to find them being replaced by VR-videos, may cost a lot. However the additionally needed development will not be too great, if the software is developed in a scalable and adaptable manner.

\section{Competition}
The competition is the biggest issue concerning this idea. To enter the market, the version control software should be integrated into a video editing application and the producers of these applications are gigantic sophisticated companies, like Adobe, Microsoft and Sony. Competing against these companies would be very tough and costly. It would not only be difficult to develop similarly sophisticated video editing software, but also to gain customers. Since learning how to use video editing software is a long process and not all software is the same, users are often reluctant to change.
A simple way around this conundrum is to work together with existing companies. The version control system can either be integrated into the software or later added as a plugin\footnote{Where can you reduce factors and not reduce value?}. This would change the competitors to other third party developers for video editing software.

\section{Value Innovation}
The version control system will be integrated into existing video editing tools. The user has the choice between a software with our version control or the same software without it. Assuming, that our version control does not cause a big overhead, no value will be lost for the customer. However value will be gained. First of all the amount of needed disk space will be reduced, thus reducing the overall cost of editing. It also eases organization of video data, potentially increasing the editing speed. On top of that the editor has the opportunity to use an older version, which he would otherwise not have cared to save. In total the video data version control system only gives advantages to the user\footnote{What new factors can you create?}.
It would be possible to compare my idea to existing additions to video editing software. This however would always require a specific case analysis, since these additions differ strongly in what they do and how they do it. Moreover no additions do what my idea will do, making a comparison ever harder\footnote{Which factors can you raise above competitors?}.

\section{Opportunity Identification}
Making films / videos generates a lot of data. This brings up many issues, the most obvious one is the storage. For every bit that memory is becoming cheaper, videos use more data. The amount of daily generated videos is steadily increasing and the videos are getting bigger in size. The standard resolution has gone up from 1k to 2k in no time. In many commercials 4k is already being advertised. Additionaly, with the growing popularity of 3D-Cinema, the amount of saved data doubles\footnote{What evidence can you provide that your proposed problem is real?}. This mass in data for the film industry alone makes organization ever harder.
A version control software would promise to take care of both of these issues, saving the film industry a lot of money.

\section{First Planning Outlook}
Virtual reality poses a real threat to this business idea. If virtual reality videos replace normal videos, the standard filetype of videos will also change. This will make our video version control software useless, except if the correct measures are taken during development. During investigation of the method that will be used to realize the software, the dependence on any particular filetype should be minimized. Also to prevent companies who imitate our software from gaining an advantage over us, long lasting contracts with the big players Adobe, Microsoft and Sony have to be formed\footnote{How will you make your company and product advantage superior and sustainable? }.
To build my team I am already getting to know other engineers, who are also fascinated about the topic of digital media. During my studies I plan to meet people who will remain my friends and also potentially my business partners\footnote{What steps will you take to build the right team at the right time? }.

\section{What is the diffusion pattern of your technology? In which phase are you?}
To answer this question we have to first decide which technology we are talking about. The idea consists of two parts. Firstly the version control, secondly a video codec. Since we are not developing anything new to the principles of version control, we should be talking about the other technology: The video codec. The video codec as such is already at the end of the diffusion pattern. Video codecs are a widely used tool, which everybody already relies on. Video codecs are however not all the same. The MPEG Codec is surely the most widely used, where as not many people have even heard of the MJP2 Codec. If we are to develop a new codec then it is easy to assume, that it will end up where all codecs end up. Namely in the adaption phase. For our codec this will be different, since it already has a useful application. Right now however we are still in the very beginning of the innovation phase.
 

\theendnotes

\end{document}