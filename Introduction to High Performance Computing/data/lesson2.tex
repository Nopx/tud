\documentclass[a4paper]{article}
\usepackage{fullpage}
\oddsidemargin = -0.5in
\newcommand\tab[1][1cm]{\hspace*{#1}}

\iffalse added for extra functionality \fi
\usepackage{booktabs}% http://ctan.org/pkg/booktabs
\newcommand{\tabitem}{~~\llap{\textbullet}~~}
\newcommand{\arrow}{$\xrightarrow[]{}$}
\usepackage{graphicx,wrapfig,lipsum,mathtools}


\begin{document}

\section{Parallel computing}
Doing things concurrently $\cong$ more complex than sequential programming. Central questions:
\begin{itemize}
\setlength{\itemsep}{-4pt}
\item BASICS: How do we split up tasks?
\item BASICS: How do the different parts communicate?
\item PERFORMANCE: How much time does it take to run the application? (User's question)
\item PERFORMANCE: Is the system fully utilized? (Adim's question)
\end{itemize}
A few important termas are:\\
\begin{tabular}{rl}
Computational Model:&Model that can be solved with a computer\\
Model of Computation:&The description of how to solve a Computational Model with a computer\\
Concurrent:&Parallel / At the same time\\
\end{tabular}
Typical steps in development are: \\
Specification \arrow Design \arrow Implementation \arrow Adaption to system \arrow Adaption to Machine

\subsection{Invisible Parallelism}
This is parallelism that we don't see while programming like:
\begin{itemize}
\setlength{\itemsep}{-4pt}
\item In compiler for the hardware and the OS
\item Pipelining
\item On the Bit-/Instruction-/Memory-Level
\end{itemize}

\subsection{Amdahl's Law}
Describes how much a program can be sped up:\\
\tab {\LARGE S=$\frac{1}{s+(1-s)/p}\geq\frac{1}{s}$\\}
\begin{tabular}{rl}
S:&speed up\\
s:&time used for sequential part\\
1-s:&time used for parallel part\\
p:&number of processors
\end{tabular}

\subsection{Granularity}
\begin{itemize}
\setlength{\itemsep}{-4pt}
\item Splitting up processes into pieces
\item How can they communicate? How can I combine them to run on 1 processor?
\item How small can and should I make them?
\end{itemize}

\subsection{Data Locality}
\begin{itemize}
\setlength{\itemsep}{-4pt}
\item How to minimize data retrieval time?
\item How to keep data as local as possible?
\end{itemize}

\subsection{Load Imbalane}
\begin{itemize}
\setlength{\itemsep}{-4pt}
\item Trying to balance load across processors
\item Balancing the load of regular and irregular processes
\end{itemize}

\section{Parallel Algorithms}
How do you identify parallelism?\\
\begin{itemize}
\setlength{\itemsep}{-4pt}
\item Look up algorithms
\item Look at code
\item Look at data structures
\end{itemize}

\subsection{Design}
\begin{enumerate}
\setlength{\itemsep}{0pt}
\item Decompose (Machine Dependent)
	\begin{itemize}
	\setlength{\itemsep}{-4pt}
	\item What level of parallelism? (static / dynamic)
	\item How many processors?
	\item How big is the problem?
	\end{itemize}
\item Clustering (Machine Independent)
	\begin{itemize}
	\setlength{\itemsep}{-4pt}
	\item How to bring tasks together?
	\end{itemize}
\item Implement (Machine Dependent)
\item Map work to processes (Machine Dependent)
	\begin{itemize}
	\setlength{\itemsep}{-4pt}
	\item Doing it static / dynamic?
	\item Being aware of data locality and possible load imbalance.
	\end{itemize}
\end{enumerate}

\subsection{Memory}
There are different kinds of memory:\\
\begin{tabular}{rl}
Distributed:&Programmer has to manually communicate data. This is most common.\\
Shared:&Uniform location of data \arrow easier for programmer. The bus can get overloaded here.\\
Virtual Shared:&Is distributed but pretends to be shared.
\end{tabular}

\subsection{Solving Concepts}
\begin{enumerate}
\setlength{\itemsep}{0pt}
\item Farmer Worker
	\begin{itemize}
	\setlength{\itemsep}{-4pt}
	\item Farmer organizes and schedules
	\item Workers do work
	\item Farmer can push work to workers (Push-Model)
	\item Workers can pull work from farmer (Pull-Model)
	\end{itemize}
\item Divide \& Conquer
	\begin{itemize}
	\setlength{\itemsep}{-4pt}
	\item Recursive split up (for example)
	\item Hardest part is the Administration
	\item First divide problem then conquer subparts
	\end{itemize}
\item Data Parallelism
	\begin{itemize}
	\setlength{\itemsep}{-4pt}
	\item When data is splitable
	\item Different nodes work on different parts of data
	\end{itemize}
\item Task Parallelism
	\begin{itemize}
	\setlength{\itemsep}{-4pt}
	\item Different tasks for different nodes
	\item Task can start when input is available (data dependent)
	\item Task can start when its output is needed (demand-flow)
	\end{itemize}
\item Bulk Synchronisation
	\begin{itemize}
	\setlength{\itemsep}{-4pt}
	\item Every part waits at a certain point for every other part and continues together
	\end{itemize}
\end{enumerate}

\subsection{System Level Communication}
{\bf With Shared Memory}
	\begin{itemize}
	\setlength{\itemsep}{-4pt}
	\item Mutual exclusion per data (Only one can access)
	\item (Un)locking mechanisms put in place
	\item Bariers put in place where processes wait
	\end{itemize}
{\bf With Distributed Memory}
	\begin{itemize}
	\setlength{\itemsep}{-4pt}
	\item Communication has to happen
	\item After sending data, it is duplicate \arrow which is the original?
	\item General structure of send command: SEND(Buffer*, element-count, destination)
	\item General structure of receive command: RECEIVE(Buffer*, element-count, owner)
	\end{itemize}


\end{document}