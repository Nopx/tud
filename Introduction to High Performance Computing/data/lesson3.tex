\documentclass[a4paper]{article}
\usepackage{fullpage}
\oddsidemargin = -0.5in

\iffalse added for extra functionality \fi
\usepackage{booktabs}% http://ctan.org/pkg/booktabs
\newcommand{\tabitem}{~~\llap{\textbullet}~~}
\newcommand{\arrow}{$\xrightarrow[]{}$}
\usepackage{graphicx,wrapfig,lipsum,mathtools}


\begin{document}

\section{Parallel Programming Models (PM)}
The PM has to talk about the language, because they have different levels of abstraction (Java vs. Assembly)\\
{\bf Quick Introduction}\\
\begin{tabular}{rl}
Process:& Program in execution processed by processor\\&It can have 3 states: running, starting, waiting, terminated\\& For a waiting process it is saved where he will continue execution\\&Processor has Program counter to know which process' turn is next\\&There is 1 per CPU\\
Thread:& To do things parallel in a process\\& If 1 thread does IO-Block, the whole process is blocked, except if you use kernel threads\\& There is one per Core
\end{tabular}

\subsection{Programming in parallel}
First start out with a sequential program, then adapt and search for parts which can be parallelized.\\
\begin{tabular}{rl}
Control Model:& Do the same / different things in parallel\\
Data:& Can be shared / private\\
Synchronization:& Is generally done explicitly\\
Communication:& Is done via shared variables
\end{tabular}

\subsection{Open Multi Processing (OpenMP)}
Syntax: \# pragma omp directive [clauses] \\
\begin{tabular}{lll}
(e.g. &\# pragma omp parallel private (var)&creates parallel part with a private variable\\
&\# pragma omp for&creates parallel for loop\\
&\# pragma omp section&does a single execution of following code\\
&\# pragma omp for schedule (static,2)&everybody does 2 following itereations)\\
\end{tabular}\\
Mainly meant for task and data paralelism.

\subsection{pThreads}
Just normal threads, you can do anything.

\subsection{High Performance Fortran (HPF}
This is Fortran with added annotations.\\
Statements distribute data among nodes. Then something is programmed for everyone to do.\\
\arrow only data parallelism is possible
\begin{itemize}
\setlength{\itemsep}{-3pt}
\item The Control is Data Driven
\item The Synchronization happens implicitly
\end{itemize}
You can basically distribute an array over multiple nodes and run the same operation everywhere.\\
Different Arrays can be aligned with each other on different nodes. \arrow both have same elements on same node.\\
It is not possible to run different tasks on different nodes. Everybody runs the same thing.

\subsection{Message Passing Interface (MPI)}
Consists of independent processes, that each have a number. They can pass messages between each other.\\ Everything is done explicitly. For example everybody computes for part of vector and then shares.\\

\subsection{Threading Building Blocks (TBB)}
Invented by MIT, bought by Intel.\\
Extension of C++, specify task and when/ how to run it. Don't specify amount of nodes.\\
Parallel actions defined in classes. Classes then run as parallel.

\subsection{Cilk}
Extension of C with simple statements. Mainly for Divide \& Conquer.\\
Everything (including dynamic multithreading) through statements "spawn" and "sync"

\subsection{Satin}
Just like Cilk just based on Java.

\subsection{Pattern Based Languages}
Idea: A set of often used methods get implemented with templates \arrow very high abstraction.

\subsection{Map Reduce}
Never forget this.

\subsection{Comparing Programming Models}
With certain criteria:\\
{\bf \begin{tabular}{ccccccc}
 Programmability &=&Performance&+&Productivity&+&Portability
\end{tabular}}\\
Portabilities:\\
\begin{tabular}{rl}
Code Portability:&Can Code be moved to other machines\\
Parallelism Portability:& Does Parallelism work on other machines\\
Performance Portability:&Does Performance stay on other machines\\
\end{tabular}

\end{document}