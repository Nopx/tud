\documentclass[a4paper]{article}
\usepackage{fullpage}
\oddsidemargin = -0.5in

\iffalse added for extra functionality \fi
\usepackage{booktabs}% http://ctan.org/pkg/booktabs
\newcommand{\tabitem}{~~\llap{\textbullet}~~}
\newcommand{\arrow}{$\xrightarrow[]{}$}
\usepackage{graphicx,wrapfig,lipsum,mathtools}


\begin{document}

\section{Parallel Programming Models (PM)}
The PM has to talk about the language, because they have different levels of abstraction (Java vs. Assembly)\\
{\bf Quick Introduction}\\
\begin{tabular}{rl}
Process:& Program in execution processed by processor\\&It can have 3 states: running, starting, waiting, terminated\\& For a waiting process it is saved where he will continue execution\\&Processor has Program counter to know which process' turn is next\\&There is 1 per CPU\\
Thread:& To do things parallel in a process\\& If 1 thread does IO-Block, the whole process is blocked, except if you use kernel threads\\& There is one per Core
\end{tabular}

\subsection{Programming in parallel}
First start out with a sequential program, then adapt and search for parts which can be parallelized.\\
\begin{tabular}{rl}
Control Model:& Do the same / different things in parallel\\
Data:& Can be shared / private\\
Synchronization:& Is generally done explicitly\\
Communication:& Is done via shared variables
\end{tabular}

\subsection{OpenMP}
\subsection{pThreads}
\subsection{High Performance Fortran}
\subsection{MPI}

\end{document}